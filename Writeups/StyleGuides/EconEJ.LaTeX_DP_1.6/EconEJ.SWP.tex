\documentclass[12pt,a4paper,english,fleqn]{article}
%%%%%%%%%%%%%%%%%%%%%%%%%%%%%%%%%%%%%%%%%%%%%%%%%%%%%%%%%%%%%%%%%%%%%%%%%%%%%%%%%%%%%%%%%%%%%%%%%%%%%%%%%%%%%%%%%%%%%%%%%%%%%%%%%%%%%%%%%%%%%%%%%%%%%%%%%%%%%%%%%%%%%%%%%%%%%%%%%%%%%%%%%%%%%%%%%%%%%%%%%%%%%%%%%%%%%%%%%%%%%%%%%%%%%%%%%%%%%%%%%%%%%%%%%%%%
\usepackage{mathptmx}
\usepackage[scaled=0.86]{helvet}
\usepackage{courier}
\usepackage[T1]{fontenc}
\usepackage[latin9]{inputenc}
\usepackage{fancyhdr}
\usepackage{url}
\usepackage{amstext}
\usepackage{graphicx}
\usepackage[authoryear]{natbib}
\usepackage[font=footnotesize,labelfont=bf]{caption}
\usepackage[colorlinks=true,linkcolor=black,citecolor=black,urlcolor=EconomicsBlue,pdfpagemode=UseNone,pdfstartview=FitH,]{hyperref}
\usepackage{babel}
\usepackage{color}
\usepackage[hang]{footmisc}

%TCIDATA{OutputFilter=LATEX.DLL}
%TCIDATA{Version=5.50.0.2890}
%TCIDATA{<META NAME="SaveForMode" CONTENT="1">}
%TCIDATA{BibliographyScheme=BibTeX}
%TCIDATA{LastRevised=Monday, January 25, 2010 09:22:33}
%TCIDATA{<META NAME="GraphicsSave" CONTENT="32">}

\newcommand\typopath{C:/swp55/Graphics/EconEpsilon}
\makeatletter
\pagestyle{fancy}
\setcounter{secnumdepth}{2}
\definecolor{EconomicsGray}{RGB}{198,212,225}
\definecolor{EconomicsLightBlue}{RGB}{127,191,192}
\definecolor{EconomicsBlue}{RGB}{0,63,138}
\definecolor{EconomicsDarkBlue}{RGB}{0,63,117}
\fancyhf{}
\fancyhead[C]{{{\raisebox{-1.5pt}{\includegraphics[scale=.05]{\typopath}}}\hspace{-.2ex}\small{\hskip -1pt\textcolor{EconomicsDarkBlue}{\textsf{\textbf{conomics}\textmd{~Discussion~Paper}}}}}}
\renewcommand\headrulewidth{0pt}
\fancyfoot[R]{{\small \sf\thepage}}
\fancyfoot[L]{{\small\sf{{www.economics-ejournal.org}}}}
\renewcommand\headheight{17pt}
\renewcommand\section{\@startsection{section}{1}{\z@}
{-5ex \@plus -1ex \@minus -.2ex}{2.3ex \@plus.2ex}
 {\normalfont\large\bf}}
\renewcommand\subsection{\@startsection{subsection}{2}{\z@}
{-3.25ex\@plus -1ex \@minus -.2ex}{1.5ex \@plus .2ex}{\normalfont\bf}}
\renewcommand{\hangfootparindent}{1em}
\renewcommand{\hangfootparskip}{0em}
\renewcommand{\footnotemargin}{0.00001pt}
\def\footnotelayout{\hspace{1em}}
\tolerance 1414
\hbadness 1414
\emergencystretch 1.5em
\hfuzz 0.3pt
\widowpenalty = 10000
\clubpenalty=10000
\vfuzz \hfuzz
\raggedbottom
\def\UrlFont{\normalfont}
\makeatother
\input{tcilatex}
\begin{document}

\date{}
\title{\noindent Here is the Title of Your Paper}
\author{Your name\thanks{%
Give your affiliation and thanks here.}.}
\maketitle

\begin{abstract}
\noindent Here comes your abstract.

This is a template for documents prepared with Scientific Workplace 5.5 for
the journal Economics-The Open Access, Open Assessment E-Journal
(www.economics-ejournal.org).

You find further instructions in the Introduction. Please read it.

\noindent \medskip{}

\noindent \emph{Keywords: }Insert the keywords here, separated by commata.

\noindent \emph{Journal of Economic Literature Classification: }\ Insert the
classification numbers, such as \emph{\ J7, B54, }etc. here.
\end{abstract}

\noindent \newpage

\section*{\noindent}

\section{Here the Title of Your Introduction}

\noindent Here the text of your introduction

Some further notes:

This template is developed for Scientific Workplace 5.5 and has not been
tested on other versions.

For proper output, you need the graphic \texttt{EconEpsilon.jpg}, available
at \texttt{www.economics-ejournal.org}. Place it in the graphics directory
of your SWS installation \texttt{C:\TEXTsymbol{\backslash}swp55\TEXTsymbol{%
\backslash}Graphics\TEXTsymbol{\backslash}}. If you place the graphics
somewhere else, you have to change the first line in the preamble
accordingly. Currently it is\newline
\texttt{\TEXTsymbol{\backslash}newcommand\TEXTsymbol{\backslash}%
typopath\{C:/swp55/Graphics/EconEpsilon\}}.

If the file \texttt{EconEpsilon.jpg} is located in \texttt{C:\TEXTsymbol{%
\backslash}Documents\TEXTsymbol{\backslash}New}, the line has to be changed
into\newline
\texttt{\TEXTsymbol{\backslash}newcommand\TEXTsymbol{\backslash}%
typopath\{C:/Documents/New/EconEpsilon\}},\newline
for instance.\ (Note that you have to replace backslashes by slashes in the
path. Avoid blanks in paths and filenames, as LaTeX may not swallow them. I
have not tested that with SWP, though.)

Make sure that you compile with pdflatex. (View and print as PDF).

For the \ bibliograhy, use the Economics E-Journal stylesheet \texttt{%
EconEJ.bst}, available at \texttt{www.economics-ejournal.org}. In order to
make it available, you have to copy this stylesheet to the BibTeX directory
of your Scientific Workplace installation. Refer to SWP Help \ for
instructions on how to handle BibTeX style files and to create
bibliographies.)

For pictures and tables, right-click on the object and use float placement
on top of page. Remove all other ticks.


\section{Here the Title of Your Second Section}

Here your second Section, \ and so on ...

\bigskip .
\bibliographystyle{EconEJ}
\bibliography{070417_bailout,EconEJ}

\end{document}
